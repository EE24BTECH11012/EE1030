%iffalse
\let\negmedspace\undefined
\let\negthickspace\undefined
\documentclass[journal,12pt,onecolumn]{IEEEtran}
\usepackage{cite}
\usepackage{amsmath,amssymb,amsfonts,amsthm}
\usepackage{algorithmic}
\usepackage{graphicx}
\usepackage{textcomp}
\usepackage{xcolor}
\usepackage{txfonts}
\usepackage{listings}
\usepackage{enumitem}
\usepackage{mathtools}
\usepackage{gensymb}
\usepackage{comment}
\usepackage[breaklinks=true]{hyperref}
\usepackage{tkz-euclide} 
\usepackage{listings}
\usepackage{gvv}                                        
%\def\inputGnumericTable{}                                 
\usepackage[latin1]{inputenc}                                
\usepackage{color}                                            
\usepackage{array}                                            
\usepackage{longtable}                                       
\usepackage{calc}   
\usepackage{multicol}
\usepackage{multirow}                                         
\usepackage{hhline}                                           
\usepackage{ifthen}                                           
\usepackage{lscape}
\usepackage{tabularx}
\usepackage{array}
\usepackage{float}


\newtheorem{theorem}{Theorem}[section]
\newtheorem{problem}{Problem}
\newtheorem{proposition}{Proposition}[section]
\newtheorem{lemma}{Lemma}[section]
\newtheorem{corollary}[theorem]{Corollary}
\newtheorem{example}{Example}[section]
\newtheorem{definition}[problem]{Definition}
\newcommand{\BEQA}{\begin{eqnarray}}
\newcommand{\nCr}[2]{\,^{#1}C_{#2}}
\newcommand{\EEQA}{\end{eqnarray}}
\newcommand{\define}{\stackrel{\triangle}{=}}
\theoremstyle{remark}
\newtheorem{rem}{Remark}

% Marks the beginning of the document
\begin{document}
\bibliographystyle{IEEEtran}
\vspace{3cm}

\title{JEE Questions 3}
\author{EE24BTECH11012 - Bhavanisankar G S}
\maketitle
\newpage
\bigskip

\renewcommand{\thefigure}{\theenumi}
\renewcommand{\thetable}{\theenumi}
\begin{enumerate}
	\item Let $\vec{A}$ and $\vec{B}$ be $3 \times 3$ real matrices such that $\vec{A}$ is symmetric matrix and $\vec{B}$ is skew-symmetric matrix. Then the sytem of linear equations $\brak{\vec{A^2B^2 - B^2A^2}}\vec{X} = \vec{O}$, where $\vec{X}$ is a $3 \times 1$ column matrix of unknown variables and $\vec{O}$ is a $3 \times 1$ null matrix, has: 
		\begin{enumerate}
			\item a unique solution
			\item exactly two solutions
			\item infinitely many solutions
			\item no solution
		\end{enumerate}
	\item If $ n\geq2 $ is a positive integer, then the sum of the series $\nCr{n+1}{2} + 2\brak{\nCr{2}{2} + \nCr{3}{2} + \nCr{4}{2} + \dots + \nCr{n}{2}}$ is 
		\begin{enumerate}
				\begin{multicols}{4}
				\item $ \frac{n\brak{n+1}^2\brak{n+2}}{12}$
				\item $ \frac{n\brak{n-1}\brak{2n+1}}{6} $
				\item $ \frac{n\brak{n+1}\brak{2n+1}}{6} $
				\item $ \frac{n\brak{2n+1}\brak{3n+1}}{6} $
				\end{multicols}
		\end{enumerate}
	\item If a curve $ y = f\brak{x} $ passes through the point $\myvec{1,2}$ and satisfies $ x\frac{dy}{dx} + y = \emph{b}x^4 $, then for what value of \emph{b}, $ \int_{1}^{2} f\brak{x} dx = \frac{62}{5} $ holds good ?
		\begin{enumerate}
				\begin{multicols}{4}
			\item 5
			\item $\frac{62}{5}$
			\item $\frac{31}{5}$
			\item 10
				\end{multicols}
		\end{enumerate}
	\item The area of the region: $ \textbf{R}\cbrak{\myvec{x,y} : 5x^2 \leq y \leq 2x^2 + 9 } $ is :
		\begin{enumerate}
				\begin{multicols}{4}
				\item $ 9\sqrt3 $
				\item $ 12\sqrt3 $
				\item $ 11\sqrt3 $
				\item $ 6\sqrt3 $
				\end{multicols}
		\end{enumerate}
	\item Let f\brak{x} be a differentiable function defined on \sbrak{0,2} such that $ f^{\prime}\brak{x} = f^{\prime}\brak{2-x} $ for all x $\in$ \brak{0,2}, $ f\brak{0} = 1 $ and $ f\brak{2} = e^2 $ . Then the value of $ \int_{0}^{2} f\brak{x} dx $ is: 
		\begin{enumerate}
				\begin{multicols}{4}
				\item $ 1+e^2 $
				\item $ 1-e^2 $
				\item $ 2\brak{1-e^2} $
				\item $ 2\brak{1+e^2} $
				\end{multicols}
		\end{enumerate}
\end{enumerate}

		\section{Integer-Type Questions}
\begin{enumerate}
	\item The number of real roots of the equation $ \brak{x+1}^2 + \abs{x-5} = \frac{27}{4} $ is : 
	\item The students $S_1, S_2, \dots, S_{10}$ are to be divided into 3 groups A, B and C such that each group has at least one student and the group C has at most 3 students. Then the total number of possibilities of forming such groups is :
	\item If $ \emph{a} + \alpha = 1 $, $ \emph{b} + \beta = 2 $ and $af\brak{x} + \alpha\brak{1}{x} = bx + \frac{\beta}{2} $, $ x\neq0 $ then the value of the expression $\frac{\sbrak{f\brak{x} + f\brak{\frac{1}{x}}}}{\brak{x + \frac{1}{x}}}$ :
	\item If the variance of 10 natural numbers 1,1,1,\dots,1,\emph{k} is less than 10, then the maximum possible value of \emph{k} is :
	\item Let $\lambda$ be an integer. If the shortest distance between the lines $ x - \lambda = 2y - 1 = = -2z $ and $ x = y + 2\lambda = z - \lambda $ is $\frac{\sqrt{7}}{2\sqrt{2}}$, then the value of $\abs{\lambda}$ is :
	\item If $ \emph{i} = \sqrt{-1} $. If $ \frac{\brak{-1+\emph{i}\sqrt{3}}^21}{\brak{1-\emph{i}}^24} + \frac{\brak{1+\emph{i}\sqrt{3}}^21}{\brak{1+\emph{i}}^24} = \emph{k}$, and $ \emph{n} = \sbrak{\abs{\emph{k}}}$ be the greatest integral part of $\abs{\emph{k}}$. Then $ \sum_{j=0}^{\emph{n}+5} \brak{j+5}^2 - \sum_{j=0}^{\emph{n+5}} \brak{j+5} $ is equal to :
	\item Let a point $\vec{P}$ be such that its distance from the point $\myvec{5,0}$ is thrice the distance of $\vec{P}$ from the point $\myvec{-5,0}$. If the locus of the point $\vec{P}$ is a circle of radius \emph{r}, then $4\emph{r}^2$ is equal to :
	\item The maximum value of k for which the sum $ \sum_{i=0}^{k} \nCr{10}{i} \nCr{15}{k-i} + \sum_{i=0}^{k+1} \nCr{12}{i} \nCr{13}{k+1-i} $ exists, is equal to :

	\item The sum of first four terms of a geometric progression is $\frac{65}{12}$ and the sum of their respective reciprocals is $\frac{65}{18}$. If the product of first three terms of the G.P. is 1, and the third term is $\alpha$ then $2\alpha$ is :

	\item If the area of the triangle formed by the positive x-axis, the normal and the tangent to the circle $\brak{x-2}^2 + \brak{y-3}^2 = 25$ at the point $\myvec{5,7}$ is \emph{A}, then 24\emph{A} is equal to :
\end{enumerate}
 \end{document}
