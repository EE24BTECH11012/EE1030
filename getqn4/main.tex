\let\negmedspace\undefined
\let\negthickspace\undefined
\documentclass[journal]{IEEEtran}
\usepackage[a5paper, margin=10mm, onecolumn]{geometry}
%\usepackage{lmodern} % Ensure lmodern is loaded for pdflatex
\usepackage{tfrupee} % Include tfrupee package

\setlength{\headheight}{1cm} % Set the height of the header box
\setlength{\headsep}{0mm}     % Set the distance between the header box and the top of the text

\usepackage{gvv-book}
\usepackage{gvv}
\usepackage{cite}
\usepackage{amsmath,amssymb,amsfonts,amsthm}
\usepackage{algorithmic}
\usepackage{graphicx}
\usepackage{textcomp}
\usepackage{xcolor}
\usepackage{txfonts}
\usepackage{listings}
\usepackage{enumitem}
\usepackage{mathtools}
\usepackage{gensymb}
\usepackage{comment}
\usepackage[breaklinks=true]{hyperref}
\usepackage{tkz-euclide} 
\usepackage{listings}
% \usepackage{gvv}                                        
\def\inputGnumericTable{}                                 
\usepackage[latin1]{inputenc}                                
\usepackage{color}                                            
\usepackage{array}                                            
\usepackage{longtable}                                       
\usepackage{calc}                                             
\usepackage{multirow}                                         
\usepackage{hhline}                                           
\usepackage{ifthen}                                           
\usepackage{lscape}
\begin{document}

\bibliographystyle{IEEEtran}
\vspace{3cm}

\title{1.9.15}
\author{EE24BTECH11012 - Bhavanisankar G S}
% \maketitle
% \newpage
% \bigskip
{\let\newpage\relax\maketitle}

\renewcommand{\thefigure}{\theenumi}
\renewcommand{\thetable}{\theenumi}
\setlength{\intextsep}{10pt} % Space between text and floats


\numberwithin{equation}{enumi}
\numberwithin{figure}{enumi}
\renewcommand{\thetable}{\theenumi}

\textbf{QUESTION} \\
If $\vec{a}$, $\vec{b}$, $\vec{c}$ are position vectors of the points $\vec{A}$ \myvec{2 \\ 3 \\ -4}, $\vec{B}$ \myvec{ 3 \\ -4 \\ -5 }, and $\vec{C}$ \myvec{ 3 \\ 2 \\ -3} respectively, then $\abs{\vec{a}+\vec{b}+\vec{c}}$ is equal to \\
\textbf{SOLUTION} \\
\begin{table}[h!]
	\centering
        \begin{tabular}[12pt]{|c|c|c|}
    \hline
    \textbf{Variable name} & \textbf{Description} & \textbf{Formula}\\ 
    \hline
		A & $\vec{2,3,-4}$ &  $\abs{\vec{a,b,c}}=\sqrt{a^2+b^2+c^2}$\\
    \hline 
                B & $\vec{3,-4,-5}$ & $\abs{\vec{A+B+C}}$ = ? \\
    \hline
		C & $\vec{3,2,-3}$ & .  \\
    \hline   
\end{tabular}


	\caption{Variables Used}
	\label{tab10.5.3.9.1}
\end{table} \\ \\ \\
Let 
\begin{align}
	\vec{A}+\vec{B}+\vec{C} &= \myvec{8\\1\\-12}  \\
	\abs{\myvec{a\\b\\c}} &= \sqrt{a^2 + b^2 + c^2} \\
	\implies	\abs{\vec{a}+\vec{b}+\vec{c}} &= \sqrt{209}  \\
\end{align}
Hence, the answer to the given question is $\sqrt{209}$.

\begin{figure}[h]
	\centering
	\includegraphics[width=0.8\textwidth]{figs/figure.jpg}
	\caption{A plot of the points given with the origin}
\end{figure}
\end{document}
