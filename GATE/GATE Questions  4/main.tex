%iffalse
\let\negmedspace\undefined
\let\negthickspace\undefined
\documentclass[journal,12pt,onecolumn]{IEEEtran}
\usepackage{cite}
\usepackage{amsmath,amssymb,amsfonts,amsthm}
\usepackage{algorithmic}
\usepackage{graphicx}
\usepackage{textcomp}
\usepackage{xcolor}
\usepackage{txfonts}
\usepackage{listings}
\usepackage{enumitem}
\usepackage{mathtools}
\usepackage{gensymb}
\usepackage{comment}
\usepackage[breaklinks=true]{hyperref}
\usepackage{tkz-euclide} 
\usepackage{listings}
\usepackage{gvv}                                        
%\def\inputGnumericTable{}                                 
\usepackage[latin1]{inputenc}                                
\usepackage{color}                                            
\usepackage{array}                                            
\usepackage{longtable}                                       
\usepackage{calc}   
\usepackage{multicol}
\usepackage{multirow}                                         
\usepackage{hhline}                                           
\usepackage{ifthen}                                           
\usepackage{lscape}
\usepackage{tabularx}
\usepackage{array}
\usepackage{float}
\usepackage{circuitikz}

\newtheorem{theorem}{Theorem}[section]
\newtheorem{problem}{Problem}
\newtheorem{proposition}{Proposition}[section]
\newtheorem{lemma}{Lemma}[section]
\newtheorem{corollary}[theorem]{Corollary}
\newtheorem{example}{Example}[section]
\newtheorem{definition}[problem]{Definition}
\newcommand{\BEQA}{\begin{eqnarray}}
\newcommand{\nCr}[2]{\,^{#1}C_{#2}}
\newcommand{\EEQA}{\end{eqnarray}}
\newcommand{\define}{\stackrel{\triangle}{=}}
\theoremstyle{remark}
\newtheorem{rem}{Remark}

% Marks the beginning of the document
\begin{document}
\bibliographystyle{IEEEtran}
\vspace{3cm}

\title{GATE Questions 4}
\author{EE24BTECH11012 - Bhavanisankar G S}
\maketitle
\newpage
\bigskip

\renewcommand{\thefigure}{\theenumi}
\renewcommand{\thetable}{\theenumi}
\begin{enumerate}
	\item Water flows from an open vertical cylindrical tank of 20 cm diameter through a hole of 10 cm diameter. What will be the velocity of water flowing out of the hole at the instant when the water level in the tank is 50 cm above the hole ? Ignore unsteady effects.
		\begin{enumerate}
				\begin{multicols}{4}
				\item 3.16 m/s
				\item 3.26 m/s
				\item 3.36 m/s
				\item 3.46 m/s
				\end{multicols}
		\end{enumerate}
	\item In the manometer shown in the figure, the pressure $p_A$ of the gas inside bulb A is approximately
		\begin{figure}[H]
			\centering
			\begin{circuitikz}
\tikzstyle{every node}=[font=\small]
\draw  (2,15.5) circle (0.5cm);
\draw  (2,15.5) circle (0.5cm);
\draw [short] (1.75,15) -- (1.75,12.5);
\draw [short] (2.25,15) -- (2.25,13);
\draw [short] (2.25,13) -- (4.25,13);
\draw [short] (1.75,12.5) -- (4.75,12.5);
\draw [short] (4.25,13) -- (4.25,15.25);
\draw [short] (4.75,12.5) -- (4.75,15.25);
\draw [dashed] (1.75,14) -- (2.25,14);
\draw [dashed] (1.75,13.75) -- (2.25,13.75);
\draw [dashed] (1.75,13.5) -- (2.25,13.5);
\draw [dashed] (1.75,13.25) -- (2.25,13.25);
\draw [dashed] (1.75,13) -- (2.25,13);
\draw [dashed] (1.75,12.75) -- (4.75,12.75);
\draw [dashed] (1.75,13) -- (4.75,13);
\draw [dashed] (1.75,12.75) -- (4.75,12.75);
\draw [dashed] (1.75,13) -- (2.75,13);
\draw [dashed] (1.75,12.75) -- (4.5,12.5);
\draw [dashed] (1.75,12.5) -- (4.5,12.75);
\draw [dashed] (4.25,13.25) -- (4.75,13.25);
\draw [short] (4.25,14) -- (5,14.5);
\draw [short] (4.25,13.75) -- (4.75,14);
\draw [short] (4.25,13.5) -- (4.75,13.75);
\draw [short] (4.25,14.5) -- (4.75,14.25);
\draw [short] (4.25,14.5) -- (4.75,14);
\draw [short] (4.25,14.25) -- (4.75,13.75);
\draw [short] (4.25,14) -- (4.75,13.5);
\draw [short] (4.25,13.75) -- (4.75,13.25);
\node [font=\small] at (2,15.5) {$p_A$};
\node [font=\small] at (1.25,15.5) {A};
\node [font=\small] at (2.5,14.5) {20 cm};
\node [font=\small] at (5.25,14) {30 cm};
\node [font=\small] at (4.5,15.75) {$P_{atm} = 1bar$};
\node [font=\small] at (3.75,14.25) {Hg};
\node [font=\small] at (3,12.25) {Water};
\draw [<->, >=Stealth] (3.25,14.25) -- (3.25,15);
\draw [<->, >=Stealth] (6,13.25) -- (6,14.75);
\end{circuitikz}

			\caption{}
			\label{25}
		\end{figure}
		\begin{enumerate}
				\begin{multicols}{4}
				\item 0.8 bar
				\item 1.2 bar
				\item 1.4 bar
				\item 1.6 bar
				\end{multicols}
		\end{enumerate}
	\item Consider a fully developed laminar flow in a circular pipe. If the diameter of the pipe is halved while the flow rate and length of the pipe are kept constat, the head loss increase by a factor of
		\begin{enumerate}
				\begin{multicols}{4}
				\item 4
				\item 8
				\item 16
				\item 32
				\end{multicols}
		\end{enumerate}
	\item A 1:20 model of a sub-marine is to be tested in a towing tank containing sea water. If the sub-marine velocity is 6 m/s, at what velocity should the model be towed for dynamic similarity ?
		\begin{enumerate}
				\begin{multicols}{4}
				\item 60 m/s
				\item 120 m/s
				\item 180 m/s
				\item 240 m/s
				\end{multicols}
		\end{enumerate}
	\item An oil droplet ( density = $800 kg/(m^2)$ ) is rising in still water at a constant veloxity of 1 mm/s. Its radius is approximately
		\begin{enumerate}
				\begin{multicols}{4}
				\item $ 21 \mu $
				\item $ 24 \mu $
				\item $ 34 \mu $
				\item $ 47 \mu $
				\end{multicols}
		\end{enumerate}
	\item Determine the correctness or otherwise of the following \textbf{Assertion} and \textbf{Reason} \\
		\textbf{Assertion}: The coefficient of discharge of orifice flow meter is less than that of venturi meter. \\
		\textbf{Reason}: Orifice flow meter is a differential pressure device.
		\begin{enumerate}
			\item Both A and R are true and R is the correct reason for A
			\item Both A and R are true but R is not the correct explanation of A
			\item Both A and R are false
			\item A is true but R is false.
		\end{enumerate}\\
\section{Common Data Questions} \\
		A long cylindrical object submerged in still water is moving at constant speed of 5 m/s perpendicular to its axis, as shown in the figure. Neglect viscous effects and assume free stream pressure to be 100 kPa.
		\begin{figure}[H]
			\centering
			\begin{circuitikz}
\tikzstyle{every node}=[font=\small]
\draw  (5,12.75) circle (1.75cm);
\draw [dashed] (1.5,12.75) -- (8.75,12.75);
\draw [short] (5,12.75) -- (7,14.5);
\draw [->, >=Stealth] (5,12.75) -- (2.75,12.75);
\node [font=\small] at (6.5,14) {P};
\node [font=\small] at (5.75,13) {45 $\circ$};
\node [font=\small] at (2.5,12.5) {5 m/s};
\node [font=\small] at (5,12.5) {O};
\end{circuitikz}

			\caption{}
			\label{25}
		\end{figure}
	\item The fluid velocity at point P with respect to the cylinder will be approximately
		\begin{enumerate}
				\begin{multicols}{4}
				\item 3.5 m/s
				\item 5 m/s
				\item 7 m/s
				\item 10 m/s
				\end{multicols}
		\end{enumerate}
	\item The absolute pressure at point P will be approximately
		\begin{enumerate}
				\begin{multicols}{4}
				\item 137 kPa
				\item 112 kPa
				\item 87 kPa
				\item 62 kPa
				\end{multicols}
		\end{enumerate} \\
	The velocity field for a two dimensional flow is given by 
	$$ \overline{V} \brak{x,y,t} = \frac{x}{t} \vec{i} - \frac{y}{t} \vec{j} $$
	\item The total acceleration is
		\begin{enumerate}
				\begin{multicols}{2}
				\item $ \frac{x}{t^2} \vec{i} - \frac{y}{t^2} \vec{j} $
				\item $ \frac{-x}{t^2} \vec{i}+ \frac{y}{t^2} \vec{j} $
				\item $ \frac{2x}{t^2} \vec{i} $
				\item $ \frac{2y}{t^2} \vec{j} $
				\end{multicols}
		\end{enumerate}
	\item The given velocity field is
		\begin{enumerate}
				\begin{multicols}{2}
			\item incompressible and rotational
			\item compressible and rotational
			\item incompressible and irrotational
			\item compressible and irrotational
				\end{multicols}
		\end{enumerate}
		An incompressible fluid is passed through a T-junction supported on wheels, as shown in the figure. The area of outlet A is twice that of outlet B. While the incoming mass flow rate is fixed, the distribution of flow at the two outlets can be varied by a suitable mechanism built in system. Assume tht the flexible tube offers no resistance to motion, and frictional effects in the pipes and wheels can be neglected. Now, consider the following two cases. \\
		\textbf{Case 1:} The flow rates at sections A and B are equal.
		\textbf{Case 2:} The velocities at sections A and B are eqial.
		\begin{figure}[H]
			\centering
			\begin{circuitikz}
\tikzstyle{every node}=[font=\small]
\draw (2.75,11.75) to[short] (8.5,11.75);
\draw (2.75,11.75) to[short] (8.5,11.75);
\draw (2.75,11.75) to[short] (8.5,11.75);
\draw (3.5,12.25) to[short] (7.75,12.25);
\draw (3.5,13.25) to[short] (5,13.25);
\draw (5,13.25) to[short] (5,14.25);
\draw (6.5,14.25) to[short] (6.5,13);
\draw (6.5,13.5) to[short] (6.5,12.75);
\draw (6.5,12.75) to[short] (7.75,12.75);
\draw  (4.5,12) circle (0.25cm);
\draw  (7.25,12) circle (0.25cm);
\draw [->, >=Stealth] (5.75,14.25) -- (5.75,13.5);
\draw [->, >=Stealth] (4,12.75) -- (3,12.75);
\draw [->, >=Stealth] (7.75,12.5) -- (8.5,12.5);
\draw [dashed] (5,14) -- (5,16);
\draw [dashed] (5,16) -- (7.5,16);
\draw [dashed] (6.5,14.25) -- (5.5,15.5);
\draw [dashed] (5.5,15.5) -- (7,15.5);
\draw [dashed] (7,15.5) -- (7.5,15.25);
\node [font=\small] at (7,15) {Flexible Tube};
\node [font=\small] at (4.25,12.75) {\textbf{A}};
\node [font=\small] at (7.25,12.5) {\textbf{B}};
\end{circuitikz}

			\caption{}
			\label{25}
		\end{figure}
	\item Which of the following statements are true ?\\
		\textbf{P:} In case 1, the velocity at section A is twice the velocity at section B \\
		\textbf{Q:} In case 1, the velocity at section A is half that at section B\\
		\textbf{R:} In case 2, the flow rate at section A is twice that at section B \\
		\textbf{S:} In case 2, the flow rate at section A is half that at section B \\
		\begin{enumerate}
				\begin{multicols}{4}
				\item P,R
				\item P,S
				\item Q,R
				\item Q,S
				\end{multicols}
		\end{enumerate}
	\item Which of the following statements are true ? \\
		\textbf{P:} In case 1, the system moves to the left.\\
		\textbf{Q:} In case 1, the system moves to the right.\\
		\textbf{R:} In case 2, the system moves to the left.\\
		\textbf{S:} In case 2, the system moves to the right.
		\begin{enumerate}
				\begin{multicols}{4}
				\item P,R
				\item P,S
				\item Q,R
				\item Q,S
				\end{multicols}
		\end{enumerate}

\end{enumerate}
\end{document}
