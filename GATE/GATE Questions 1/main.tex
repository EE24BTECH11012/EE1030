%iffalse
\let\negmedspace\undefined
\let\negthickspace\undefined
\documentclass[journal,12pt,onecolumn]{IEEEtran}
\usepackage{cite}
\usepackage{amsmath,amssymb,amsfonts,amsthm}
\usepackage{algorithmic}
\usepackage{graphicx}
\usepackage{textcomp}
\usepackage{xcolor}
\usepackage{txfonts}
\usepackage{listings}
\usepackage{enumitem}
\usepackage{mathtools}
\usepackage{gensymb}
\usepackage{comment}
\usepackage[breaklinks=true]{hyperref}
\usepackage{tkz-euclide} 
\usepackage{listings}
\usepackage{gvv}                                        
%\def\inputGnumericTable{}                                 
\usepackage[latin1]{inputenc}                                
\usepackage{color}                                            
\usepackage{array}                                            
\usepackage{longtable}                                       
\usepackage{calc}   
\usepackage{multicol}
\usepackage{multirow}                                         
\usepackage{hhline}                                           
\usepackage{ifthen}                                           
\usepackage{lscape}
\usepackage{tabularx}
\usepackage{array}
\usepackage{float}
\usepackage{circuitikz}

\newtheorem{theorem}{Theorem}[section]
\newtheorem{problem}{Problem}
\newtheorem{proposition}{Proposition}[section]
\newtheorem{lemma}{Lemma}[section]
\newtheorem{corollary}[theorem]{Corollary}
\newtheorem{example}{Example}[section]
\newtheorem{definition}[problem]{Definition}
\newcommand{\BEQA}{\begin{eqnarray}}
\newcommand{\nCr}[2]{\,^{#1}C_{#2}}
\newcommand{\EEQA}{\end{eqnarray}}
\newcommand{\define}{\stackrel{\triangle}{=}}
\theoremstyle{remark}
\newtheorem{rem}{Remark}

% Marks the beginning of the document
\begin{document}
\bibliographystyle{IEEEtran}
\vspace{3cm}

\title{GATE Questions 1}
\author{EE24BTECH11012 - Bhavanisankar G S}
\maketitle
\newpage
\bigskip

\renewcommand{\thefigure}{\theenumi}
\renewcommand{\thetable}{\theenumi}
\begin{enumerate}
	\item Which one of the following statements regarding the \textbf{INT} ( interrupt ) and the \textbf{BRQ} ( bus request ) pins in a CPU is true ?
	\begin{enumerate}
		\item The BRQ pin is sampled after every instruction cycle, but the INT is sampled after every machine cycle.
		\item Both INT and BRQ are sampled after every machine cycle.
		\item The INT pin is sampled after every instruction cycle, but the BRQ is sampled after every machine cycle.
		\item Both INT and BRQ are sampled after every insruction cycle.
		\end{enumerate}
	\item A bridge circuit is shown in the figure below. Which of the seqeuences given below is most suitable for balancing the bridge ?
	\begin{circuitikz} \draw
    (0,4) to[R, l=$jX_1 + R_1$] (3,4)   
    to[R, l=$R_3$] (6,4)
		-- (6,0)
		to[battery] (0,0)
		-- (0,4) 
		(0,2) to[R, l=$R_2$] (3,2)
    to[R, l=$R_4 - jX_4$] (6,2)
		-- (6,0)
		-- (0,0)
		-- (0,2)  ;                            

    	\end{circuitikz}
		\begin{enumerate}
			\item First adjust $R_4$ and then adjust $R_1$
			\item First adjust $R_2$ and then adjust $R_3$
			\item First adjust $R_2$ and then adjust $R_4$
			\item First adjust $R_4$ and then adjust $R_2$
		\end{enumerate}	
\end{enumerate}
\section{Common Data Questions}
\begin{enumerate}
	\item A three phase squirrel cage induction motor has a starting current of seven times the full load current and full load slip of 5\% .
\begin{enumerate}
	\item If an auto-transformer is used for reduced voltage starting to provide 1.5 per unit starting torque, the auto-transformation ratio \brak{\%} should be
		\begin{enumerate}
				\begin{multicols}{4}
				\item 57.77 \%
				\item 72.56 \%
				\item 78.25 \%
				\item 81.33 \%
				\end{multicols}
		\end{enumerate}
	\item If a star-delta starter is used to start this induction motor, the per unit starting torque will be
		\begin{enumerate}
				\begin{multicols}{4}
				\item 0.607
				\item 0.816
				\item 1.225
				\item 1.616
				\end{multicols}
		\end{enumerate}
	\item If a starting torque of 0.5 per unit is required then the per-unit starting current should be
		\begin{enumerate}
				\begin{multicols}{4}
				\item 4.65
				\item 3.75
				\item 3.16
				\item 2.13
				\end{multicols}
		\end{enumerate}
\end{enumerate}

	\item An indutor designed with 400 turns coil wound on an iron core of 16 cm$^2$ cross sectional area with a cut of an air gap length of 1 mm. The coil is connected to a 230 V, 50 Hz AC supply. Neglect coil resistance, core loss, iron reductance and leakage inductance.
\begin{enumerate}
	\item The current in the inductor is
		\begin{enumerate}
				\begin{multicols}{4}
				\item 18.08
				\item 9.04
				\item 4.56
				\item 2.28
				\end{multicols}
		\end{enumerate}
	\item The average force on the core to reduce the air gap will be
		\begin{enumerate}
				\begin{multicols}{4}
				\item 832.29
				\item 1666.22
				\item 3332.47
				\item 6664.84
				\end{multicols}
		\end{enumerate}
\end{enumerate}
	\item Cayley-Hamilton Theorem states that a square matrix satisfies its own characteristic equation. Consider the matrix $$ A = \myvec{-3 && 2 \\ -1 && 0 }$$
\begin{enumerate}
	\item A satisfies the relation
		\begin{enumerate}
				\begin{multicols}{2}
				\item $ A^2 + 3I + 2A^{-1} = 0 $
				\item $ A^2 + 2A + 2I = 0 $
				\item $ \brak{A+I} \brak{A+2I} = 0 $
				\item $ exp{A} = 0 $
				\end{multicols}
		\end{enumerate}
	\item $A^9$ equals
		\begin{enumerate}
				\begin{multicols}{2}
				\item 511A + 510I
				\item 309A + 104I
				\item 154A + 155I
				\item $exp{9A}$
				\end{multicols}
		\end{enumerate}
\end{enumerate}
	\item A signal is processed by a casual filter with transfer function G(S).
\begin{enumerate}
	\item For a distortion-free output signal waveform, G(s) must
		\begin{enumerate}
			\item provide zero phase shift for all frequency
			\item provide constant phase shift for all frequency
			\item provide linear phase shift that is proportional to frequency
			\item provide a phase shift that is inversely proportional to frequency
		\end{enumerate}
	\item $ G(z) = \alpha z^{-1} + \beta z^{-3}$ is a low-pass digital filter with a phase charateristics same as that of the above question if
		\begin{enumerate}
				\begin{multicols}{4}
				\item $\alpha = \beta $
				\item $\alpha = - \beta $
				\item $\alpha = \beta^{\frac{1}{3}} $
				\item $\alpha = \beta^{\frac{-1}{3}}$
				\end{multicols}
		\end{enumerate}
\end{enumerate}
	\item The associated figure shows the two types of rotate right instructions R1, R2 available in a micro-processor where Reg is a 8-digit register and C is the carry bit. The rotate left instructions L1 and L2 are similar except that C now links the most significant but of Reg instead of the least significant one.
	\begin{enumerate}
		\item Suppose Reg contains the 2's complement number 11010110.If this number is divided by 2 the answer should be
			\begin{enumerate}
					\begin{multicols}{2}
					\item 01101011
					\item 10010101
					\item 11101001
					\item 11101011
					\end{multicols}
			\end{enumerate}
		\item Such a division can be correctly performed by the following set of operations
			\begin{enumerate}
					\begin{multicols}{2}
					\item L2, R2, R1
					\item L2, R1, R2
					\item R2, L1, R1
					\item R1, L2, R2
					\end{multicols}
			\end{enumerate}
	\end{enumerate}
	\item Consider the RLC circuit shown in figure.

	\begin{circuitikz} \draw
    (0,0) to[battery] (0,4)
    to[R, l=$R-10$] (2,4)
    to[L, l=$L-1 mH$] (4,4)
    to[C, l=$C-10 \mu F$] (4,0)
    -- (0,0) ;
        \end{circuitikz}

\begin{enumerate}
	\item For a step-input $e_0$, the overshoot in the output $e_0$ will be
		\begin{enumerate}
				\begin{multicols}{2}
				\item 0, since the system is not under-damped
				\item 5 \%
				\item 16 \%
				\item 48 \%
				\end{multicols}
		\end{enumerate}
	\item If the above step response is to be observed on a non-storage CRO, then it would be best to have the $e_i$, as a
		\begin{enumerate}
				\begin{multicols}{2}
				\item step function
				\item square wave of frequency 50 Hz
				\item square wave of frequency 300 Hz
				\item square wave of frequency 2.0 kHz
				\end{multicols}
		\end{enumerate}
\end{enumerate}
	\item A 1:1 Pulse Transformer (PT) is used to trigger the SCR in the adjoint figure. The SCR is rated at 1.5 kV, 250A with $I_l = 250 mA$, $I_h = 150 mA$, and $I_{Gmax} = 150 mA$. The SCR is connected tto an inductive load, where L = 150 mH in series with a small resistance and the supply voltage is 200V DC.The forward drops of all transistors/diodes and gate-cathode junction during state are 1.0 V .
		\begin{circuitikz}
    		\draw
    		(0,0) to[diode] (0,2) 
		-- (2,2)
		to[L] (2,0)
		-- (0,0) 

			(2,0) to[L] (2,2)
			to[diode] (4,2)
			to[R, l=$R$] (6,2)
			-- (6,0)
			-- (2,0) 

			(4,2) to[diode] (4,0)
			-- (6,0)
			to[C] (6,2) ;

			
		\end{circuitikz}
		\begin{enumerate}
			\item The resistance R should be
				\begin{enumerate}
						\begin{multicols}{4}
						\item $ 4.7 k \Omega $
						\item $ 470 \Omega $
						\item $ 47 \Omega $
						\item $ 4.7 \Omega $
						\end{multicols}
				\end{enumerate}
			\item The minimum approximate volt-second ratif o f the pulse-transformer suitable for triggering the SCR should be : (volt-second rating is the maximum of the product of the voltage and width of the pulse that may be applied) 
				\begin{enumerate}
						\begin{multicols}{4}
						\item $ 2000 \mu V-s $
						\item $ 200 \mu V-s $
						\item $ 20 \mu V-s $
						\item $ 2.0 \mu V-s $
						\end{multicols}
				\end{enumerate}
		\end{enumerate}
\end{enumerate}
\end{document}
