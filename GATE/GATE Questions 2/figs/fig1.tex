\begin{tikzpicture}
\draw[->] (-1,0) -- (5,0) node[right] {$\alpha$} ;
\draw[->] (0,-1) -- (0,8) node[above] {$C_m$} ;

\draw[thick, blue] (-1,6) -- (3,0) node[above right] {$Q$} ;
\draw[thick, green] (-1,5) -- (4,0) node[above right] {$R$} ;
\draw[thick, red] (-1,3) -- (2,0) node[above right] {$P$} ;
\end{tikzpicture}
