%iffalse
\let\negmedspace\undefined
\let\negthickspace\undefined
\documentclass[journal,12pt,onecolumn]{IEEEtran}
\usepackage{cite}
\usepackage{amsmath,amssymb,amsfonts,amsthm}
\usepackage{algorithmic}
\usepackage{graphicx}
\usepackage{textcomp}
\usepackage{xcolor}
\usepackage{txfonts}
\usepackage{listings}
\usepackage{enumitem}
\usepackage{mathtools}
\usepackage{gensymb}
\usepackage{comment}
\usepackage[breaklinks=true]{hyperref}
\usepackage{tkz-euclide} 
\usepackage{listings}
\usepackage{gvv}                                        
%\def\inputGnumericTable{}                                 
\usepackage[latin1]{inputenc}                                
\usepackage{color}                                            
\usepackage{array}                                            
\usepackage{longtable}                                       
\usepackage{calc}   
\usepackage{multicol}
\usepackage{multirow}                                         
\usepackage{hhline}                                           
\usepackage{ifthen}                                           
\usepackage{lscape}
\usepackage{tabularx}
\usepackage{array}
\usepackage{float}


\newtheorem{theorem}{Theorem}[section]
\newtheorem{problem}{Problem}
\newtheorem{proposition}{Proposition}[section]
\newtheorem{lemma}{Lemma}[section]
\newtheorem{corollary}[theorem]{Corollary}
\newtheorem{example}{Example}[section]
\newtheorem{definition}[problem]{Definition}
\newcommand{\BEQA}{\begin{eqnarray}}
\newcommand{\nCr}[2]{\,^{#1}C_{#2}}
\newcommand{\EEQA}{\end{eqnarray}}
\newcommand{\define}{\stackrel{\triangle}{=}}
\theoremstyle{remark}
\newtheorem{rem}{Remark}

% Marks the beginning of the document
\begin{document}
\bibliographystyle{IEEEtran}
\vspace{3cm}

\title{JEE Questions 4}
\author{EE24BTECH11012 - Bhavanisankar G S}
\maketitle
\newpage
\bigskip

\renewcommand{\thefigure}{\theenumi}
\renewcommand{\thetable}{\theenumi}
\begin{enumerate}
	 	
 \item Let $ g(t) = \int_{\frac{- \pi}{2}}^{\frac{\pi}{2}} \cos{\brak{\frac{\pi}{4}t + f(x)}} dx$, where $f(x) = \log{\brak{x + \sqrt{x^2 + 1}}}$, x $\in \vec{R}$. Then which of the following is correct ?
	 \begin{enumerate}
			 \begin{multicols}{4}
			 \item $g(1) = g(0)$
			 \item $\sqrt{2}g(1) = g(0)$
			 \item $g(1) = \sqrt{2}g(0)$
			 \item $g(1) + g(0) = 0$
			 \end{multicols}
	 \end{enumerate}
 \item Let P be a variable point on the parabola $y=4x^2+1$.Then the locus of the mid-point of the point P and the foot of perpendicular drawn from the point P to the line $y=x$ is :
	 \begin{enumerate}
		 \item $\brak{3x-y}^2 + \brak{x-3y} + 2 = 0$
		 \item $2 \brak{3x-y}^2 + \brak{x-3y} + 2 = 0$
		 \item $\brak{3x-y}^2 + 2\brak{x-3y} + 2 = 0$
		 \item $2 \brak{x-3y}^2 + \brak{3x-y} + 2 = 0$
	 \end{enumerate}
\item The absolute value of $k$ $ \in \vec{R}$, for which the following system of linear equations 
		\begin{align}
			3x - y + 4z &= 3 \\ 
			x + 2y - 3z &= -2 \\
			6x + 5y + kz &= -3 
		\end{align}
		has infinitely many solutions is :
	\begin{enumerate}
			\begin{multicols}{4}
			\item 3
			\item -5
			\item 5
			\item -3
			\end{multicols}
	\end{enumerate}
\item If sum of the first 21 terms of the series $ \log_{9^\frac{1}{2}}{x} + \log_{9^\frac{1}{3}}{x} + \log_{9^\frac{1}{4}}{x} + \dots $, where x > 0 is 504, then x is equal to 
	\begin{enumerate}
			\begin{multicols}{4}
			\item 243
			\item 9
			\item 7
			\item 81
			\end{multicols}
	\end{enumerate}
\item In a triangle ABC, if $\abs{\vec{BC}}=3$, $\abs{\vec{CA}}=5$ and $\abs{\vec{BA}}=7$, then the projection of the vector $\vec{BA}$ on $\vec{BC}$ is equal to
	\begin{enumerate}
			\begin{multicols}{4}
			\item $\frac{19}{2}$
			\item $\frac{13}{2}$
			\item $\frac{11}{2}$
			\item $\frac{15}{2}$
			\end{multicols}
	\end{enumerate}
\end{enumerate}

\section{Integer-Type Questions}
\begin{enumerate}
	\item Let $ A = \myvec{2 & -1 & 1 \\ -1 & 2 & -1 \\ 1 & -1 & 2}$ then $det\brak{3Adj\brak{2A^{-1}}}$ is equal to :
	\item If $\myvec{\alpha , \beta}$ is a point on $y^2=6x$, that is closest to $\myvec{3,\frac{3}{2}}$ then find 2 $ \brak{\alpha+\beta} $

	\item Let a function  $g : \sbrak{0,4} \rightarrow \vec{R}$ be defined as
		$$ g(x) = \myvec{ max\brak{t^3-6t^2+9t-3}, & 0 \leq x \leq 3 \\
		                  4 - x, & 3 < x \leq 4 } $$
			then the number of points in the interval \brak{0,4} where g(x) is NOT differentiable is :
		\item The number of solutions of the equation $$\log_{x+1}{\brak{2x^2+7x+5}} + \log_{2x+5}{\brak{x+1}^2} - 4 = 0$$, x $\geq$ 0, is :

	\item Let a curve $ y = y(x)$ be givem by the solutio of the differential equation $$\cos{\brak{\frac{1}{2}\cos^{-1}{e^{-x}}}} dx = \sqrt{e^{2x} - 1} dy $$If it intersects y-axis at $y=-1$ and the intersection point of the curve with the x-axis is $\myvec{\alpha , 0}$, then $e^{\alpha}$ is equal to :
	\item For p $\geq$ 0, a vector $\vec{v_2} = 2\vec{i} + \brak{p+1}\vec{j}$ is obtained by rotating the vector $\vec{v_1} = \sqrt{3}p\vec{i} + \vec{j}$ by an angle $\theta$ about the origin in counter clockwise direction. If $\tan{\theta} = \frac{\alpha\sqrt{3} - 2}{4\sqrt{3} + 3}$, then the value of $\alpha$ is equal to :
	\item Consider a triangle with vertices $\vec{A} \myvec{-2,3}, \vec{B} \myvec{1,9}, \vec{C} \myvec{3,8}$. If a line $\vec{L}$ passing through the circumcentre of the triangle ABC, bisects line BC, and intersects y-axis at point $\myvec{0,\frac{\alpha}{2}}$ then the value of real number $\alpha$ is :
	\item For k $\in \vec{N}$, let $$ \frac{1}{\alpha(\alpha +1)(\alpha +2)\dots(\alpha +20)} = \sum_{k=0}^{20} \frac{A_k}{\alpha + k} $$ where $\alpha>0$.Then the value of 100 $\brak{\frac{A_{14} + A_{15}}{A_{13}}}^2 $ is :
	\item Let $\cbrak{a_{n}}_{n=1}^{\infty}$ be a sequence such that $a_1 = 1$, $a_2 = 1$ and $a_{n+2} = 2a_{n+1} + a_{n}$ for all $n \geq 1$. Then the value of $47 \sum_{n=1}^{\infty} \frac{a_{n}}{2^{3n}}$ is equal to :
	\item If $\lim_{x \to 0} \frac{\alpha xe^{x} - \beta \log\brak{1+x} + \gamma x^2e^{-x}}{x \sin^{2}{x}} = 10 $, $\alpha$, $\beta$, $\gamma \in \vec{R}$, then the value of $\alpha + \beta + \gamma$ is : 
\end{enumerate}
\end{document}
