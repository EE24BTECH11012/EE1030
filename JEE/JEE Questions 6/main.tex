%iffalse
\let\negmedspace\undefined
\let\negthickspace\undefined
\documentclass[journal,12pt,onecolumn]{IEEEtran}
\usepackage{cite}
\usepackage{amsmath,amssymb,amsfonts,amsthm}
\usepackage{algorithmic}
\usepackage{graphicx}
\usepackage{textcomp}
\usepackage{xcolor}
\usepackage{txfonts}
\usepackage{listings}
\usepackage{enumitem}
\usepackage{mathtools}
\usepackage{gensymb}
\usepackage{comment}
\usepackage[breaklinks=true]{hyperref}
\usepackage{tkz-euclide} 
\usepackage{listings}
\usepackage{gvv}                                        
%\def\inputGnumericTable{}                                 
\usepackage[latin1]{inputenc}                                
\usepackage{color}                                            
\usepackage{array}                                            
\usepackage{longtable}                                       
\usepackage{calc}   
\usepackage{multicol}
\usepackage{multirow}                                         
\usepackage{hhline}                                           
\usepackage{ifthen}                                           
\usepackage{lscape}
\usepackage{tabularx}
\usepackage{array}
\usepackage{float}


\newtheorem{theorem}{Theorem}[section]
\newtheorem{problem}{Problem}
\newtheorem{proposition}{Proposition}[section]
\newtheorem{lemma}{Lemma}[section]
\newtheorem{corollary}[theorem]{Corollary}
\newtheorem{example}{Example}[section]
\newtheorem{definition}[problem]{Definition}
\newcommand{\BEQA}{\begin{eqnarray}}
\newcommand{\nCr}[2]{\,^{#1}C_{#2}}
\newcommand{\EEQA}{\end{eqnarray}}
\newcommand{\define}{\stackrel{\triangle}{=}}
\theoremstyle{remark}
\newtheorem{rem}{Remark}

% Marks the beginning of the document
\begin{document}
\bibliographystyle{IEEEtran}
\vspace{3cm}

\title{JEE Questions 6}
\author{EE24BTECH11012 - Bhavanisankar G S}
\maketitle
\newpage
\bigskip

\renewcommand{\thefigure}{\theenumi}
\renewcommand{\thetable}{\theenumi}
\begin{enumerate}
	\item A plane P is parallel to two lines whose direction ratios are $\myvec{-2,1,3}$ and $\myvec{-1,2,-2}$ and it contains the point $\myvec{2,2,-2}$. Let P intersect the co-ordinate axes at the points A, B, C making the intercepts $\alpha$, $\beta$, $\gamma$. If $\vec{V}$ is te volume of the tetrahedron OABC, where O is the origin and $p=\alpha + \beta + \gamma$, then the ordered pair $\myvec{\vec{V}, p}$ is equal to :
		\begin{enumerate}
				\begin{multicols}{4}
				\item $\myvec{48,-13}$
				\item $\myvec{24,-13}$
				\item $\myvec{48,11}$
				\item $\myvec{24,-5}$
				\end{multicols}
		\end{enumerate}
	\item Let S be the set of all $ a \in \vec{R}$ for which the angle between the vectors $\vec{u} = a\brak{\log_{e}{b}}\vec{i} -6\vec{j} + 3\vec{k}$ and $\vec{v} = \log_{e}{b}\vec{i} + 2\vec{j} + 2a\log_{e}{b}\vec{k}$, $(b>1)$ is acute. Then S is equal to 
		\begin{enumerate}
				\begin{multicols}{4}
				\item $\brak{-\infty, -\frac{4}{3}}$
				\item $ \phi $
				\item $\brak{-\frac{4}{3}, 0}$
				\item $\brak{\frac{12}{7}, \infty}$
				\end{multicols}
		\end{enumerate}
	\item A horizontal park is in the shape of a triangle OAB with AB = 16. A vertical lamp post OP is erected at the point O such that $\angle{PAO} = \angle{PBO} = 15 \degree$ and $\angle{PCO} = 45\degree$, where C is the mid-point of AB. Then $\brak{OP}^2$ is equal to 
		\begin{enumerate}
				\begin{multicols}{4}
				\item $\frac{32}{\sqrt{3}}\brak{\sqrt{3}-1}$
				\item $\frac{32}{\sqrt{3}}\brak{2-\sqrt{3}}$
				\item $\frac{16}{\sqrt{3}}\brak{\sqrt{3}-1}$
				\item $\frac{16}{\sqrt{3}}\brak{2-\sqrt{3}}$
				\end{multicols}
		\end{enumerate}
	\item Let A and B be two events such that $P(B|A) = \frac{2}{5}$, $P(A|B) = \frac{1}{7}$ and $P(A \cap B) = \frac{1}{9}$. Consider \\
		\textbf{S1} : $P(A^{\prime} \cup B) = \frac{5}{6} $ \\
		\textbf{S2} : $P(A^{\prime} \cap B^{\prime}) = \frac{1}{18} $ \\
		\begin{enumerate}
			\item Both S1 and S2 are true .
			\item Both S1 and S2 are false.
			\item S1 is true, but S2 is false.
			\item S1 is false, but S1 is true.
		\end{enumerate}
	\item Let \\
		\textbf{p} : Ramesh listens to music. \\
		\textbf{q} : Ramesh is out of his village. \\
		\textbf{r} : It is Sunday. \\
		\textbf{s} : It is Saturday. \\
		Then the statement "Ramesh listens to music only if he is in his village and it is Sunday or Saturday" can be expressed as 
		\begin{enumerate}
				\begin{multicols}{2}
				\item $\brak{\brak{\neg{q}} \land \brak{r \lor s}} \implies p $
				\item $\brak{q \land \brak{r \lor s}} \implies p$
				\item $p \implies \brak{q \land \brak{r \lor s}}$
				\item $p \implies \brak{\brak{\neg{q}} \land \brak{r \lor s}}$
				\end{multicols}
		\end{enumerate}
\end{enumerate}
\section{Integer-Type Questions}
\begin{enumerate}
	\item Let the coefficients of the middle terms in the expansion of $\brak{\frac{1}{\sqrt{6}} + \beta x }^4$, $\brak{1 - 3 \beta x}^2$ and $\brak{1 - \frac{\beta}{2} x}^6$, $\brak{\beta \geq 0}$, respectively form the first three terms of an A.P. If d is the common difference of this A.P., then the value of $50 - \frac{2d}{\beta^{2}}$ is equal to :
	\item A class contains b boys and g girls. If the number of ways of selecting 3 boys and 2 girls from the class is 168, then b + 3g is equal to :
	\item Let the tangents at the points P and Q on the ellipse $\frac{x^2}{2} + \frac{y^2}{4} = 1$ meet at the point $\vec{R}\myvec{\sqrt{2}, 2\sqrt{2}-2}$. If S is the focus of the ellipse on its negative major axis, then $\brak{SP}^2 + \brak{SQ}^2$ is equal to :
	\item  If $ 1 + \brak{2 + \nCr{49}{1} + \nCr{49}{2} + \dots + \nCr{49}{49}}\brak{\nCr{50}{2} + \nCr{50}{4} + \dots + \nCr{50}{50}}$ is equal to $2^{n}m$, where m is odd, then n + m is equal to :
	\item Two tangent lines l1 and l2 are drawn from the point $\myvec{2,0}$ to the parabola $2y^2 = x$. If the lines l1 and l2 are also tangent to the circle $\brak{x-5}^2 + y^2 = r$, then 17$r$ is equal to :
	\item If $\frac{6}{3^{12}} + \frac{10}{3^{11}} + \frac{20}{3^{10}} + \frac{40}{3^{9}} + \dots + \frac{10240}{3} = 2^{n}{m}$, where $m$ is odd, then $m\cdot n$ is equal to :
	\item Let $S = \left[{-\pi, \frac{\pi}{2}}\right) - \cbrak{\frac{-\pi}{2}, \frac{-\pi}{4}, \frac{-3\pi}{4}, \frac{\pi}{4}}$. Then the number of elements in the set $$ A = \cbrak{\theta \in S : \tan{\theta}\brak{1 + \sqrt{5}\tan{2\theta}} = \sqrt{5} - \tan{2\theta}}$$ is :
	\item Let $ z = a + ib, b \neq 0$ be complex numbers satisfying $z^2 = \overline{z} 2^{1 - \abs{z}}$ Then the least value of $n \in \vec{N}$ suh that $z^{n} = \brak{z+1}^{n}$ is equal to :
	\item A bag contains  white and 6 black balls. Three balls are drawn at random from the bag. Let X be the number of white balls, among the drawn balls. If $\sigma^2$ is the variance of X, then 100 $\sigma^2$ is equal to
	\item The value of the integral $\int_{0}^{\frac{\pi}{2}} 60 \frac{\sin{6x}}{\sin{x}} dx $ is equal to :

\end{enumerate}
\end{document}
