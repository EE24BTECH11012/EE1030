\let\negmedspace\undefined
\let\negthickspace\undefined
\documentclass[journal]{IEEEtran}
\usepackage[a5paper, margin=10mm, onecolumn]{geometry}
%\usepackage{lmodern} % Ensure lmodern is loaded for pdflatex
\usepackage{tfrupee} % Include tfrupee package

\setlength{\headheight}{1cm} % Set the height of the header box
\setlength{\headsep}{0mm}     % Set the distance between the header box and the top of the text

\usepackage{gvv-book}
\usepackage{gvv}
\usepackage{cite}
\usepackage{amsmath,amssymb,amsfonts,amsthm}
\usepackage{algorithmic}
\usepackage{graphicx}
\usepackage{textcomp}
\usepackage{xcolor}
\usepackage{txfonts}
\usepackage{listings}
\usepackage{enumitem}
\usepackage{mathtools}
\usepackage{gensymb}
\usepackage{comment}
\usepackage[breaklinks=true]{hyperref}
\usepackage{tkz-euclide} 
\usepackage{listings}
% \usepackage{gvv}                                        
\def\inputGnumericTable{}                                 
\usepackage[latin1]{inputenc}                                
\usepackage{color}                                            
\usepackage{array}                                            
\usepackage{longtable}                                       
\usepackage{calc}                                             
\usepackage{multirow}                                         
\usepackage{hhline}                                           
\usepackage{ifthen}                                           
\usepackage{lscape}
\begin{document}

\bibliographystyle{IEEEtran}
\vspace{3cm}

\title{9.2.4}
\author{EE24BTECH11012 - Bhavanisankar G S}
% \maketitle
% \newpage
% \bigskip
{\let\newpage\relax\maketitle}

\renewcommand{\thefigure}{\theenumi}
\renewcommand{\thetable}{\theenumi}
\setlength{\intextsep}{10pt} % Space between text and floats


\numberwithin{equation}{enumi}
\numberwithin{figure}{enumi}
\renewcommand{\thetable}{\theenumi}

\textbf{QUESTION} \\
Find the area of the region bounded by the curves $ x^2 = 4y $, $ y=2$, $y=4$ and the y-axis in the first quadrant. \\
\textbf{SOLUTION} \\

\begin{table}[h!]
	\centering
        \begin{tabular}[12pt]{|c|c|c|}
\hline
\textbf{FORMULAE} \\
\hline
$ g(\vec{x}) = \vec{x^TVx} + 2\vec{u^Tx} + f = 0 $ \\
where, \\
$\vec{V} = \norm{n}^2\vec{I} - e^2\vec{nn^T} ,\\
\vec{u} = ce^2\vec{n} - \norm{n}^2\vec{F} ,\\
f = \norm{n}^2\norm{F}^2 - c^2e^2 $ \\
\hline
The points of intersection of the line $$ L: \vec{x} = \vec{h} + \kappa\vec{m},  \kappa \in \mathbb{R} $$ \\
with the conic section as above are given by $$ \vec{x}_i = \vec{h} + \kappa_i\vec{m} $$ \\
where \\
	$$ \kappa_i = \frac{1}{\vec{m^TVm}}\brak{\vec{-m^T\brak{Vh+u}} \pm \sqrt{\sbrak{\vec{m^T\brak{Vh+u}}^2} - g(\vec{h})\brak{\vec{m^TVm}}}} $$ \\
\hline
\end{tabular}

	\caption{Formulae Used}
	\label{tab9.2.4}
\end{table} \\ \\ \\
Substituting the given values, we have \\
\begin{align}
	\vec{V} &= \myvec{ 0 & 0 \\ 0 & 1 } \\
	\vec{u} &= \myvec{ -2 \\ 0 } \\
	f &= 0 \\
\end{align}
Substituting the values, we get the point of intersection as \\
\begin{align}
	\kappa_i &= -\myvec{1&0}\myvec{0 + 0 \\ 2 + -2} \pm \sqrt{\sbrak{\myvec{1&0}\myvec{0+0\\ 2+-2} }^2 + 4\brak{1}} \\
	\kappa_i &= 2\sqrt{2} \\
\end{align}
Hence, the point of intersection is \myvec{2\sqrt{2} \\ 2} \\
Similarly, the other point is given by \myvec{ 4 \\ 4 } .\\
The area bounded by the curve and the line is \\
\begin{align}
	\int_{2}^{4} \brak{2\sqrt{y}} dy &= \frac{4}{3}\brak{8-2\sqrt{2}} \\
	&= \frac{\brak{32 - 8\sqrt{2}}}{3} \\
\end{align}
Hence the required area is $\frac{32-8\sqrt{2}}{3}$ .
	
\begin{figure}[h]
	\centering
	\includegraphics[width=0.8\textwidth]{figs/fig.jpg}
	\caption{A plot of the given question.}
\end{figure}
\end{document}
